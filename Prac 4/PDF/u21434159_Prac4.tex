\documentclass{article}
\usepackage{graphicx}
\usepackage[fleqn]{amsmath}
\usepackage[a4paper, margin=2cm]{geometry}
\newcommand{\comment}[1]{}
\begin{document}
\title{Practical 4}
\author{Janco Spies, u21434159}
\maketitle
\subsection*{Task 1}
\begin{itemize}
    \item[1.1 #3]The error produced by the program is an Arithmetic exception at line 19. The function arguments were a=-2, b=0.
    \item[1.1 #5]The error occurs in the \textit{improve} function that is called by the main function at line 13.
    \item[1.1 #8]\textit{highest} = 0
    \item[1.1 #9]The error occurs because the \textit{improve} function is called with the arguments \textit{a}=-2 and \textit{highest}=0. This causes a division by 0. 
\end{itemize}
\begin{itemize}
    \item[1.2 #3]Process ID: 2175
    \item[1.2 #4]Invalid write of size 4.
    \item[1.2 #5]It tells me that the error comes from the \textit{capture} method at line 4 which is called by the \textit{main} method at line 9.
    \item[1.2 #6]An integer is 4 bytes long and an array of 10 integers has not been freed. Therefore 40 bytes have been lost.
    \item[1.2 #7]I would deallocate the integer array that is created in the \textit{capture} method at line 3.
\end{itemize}
\subsection*{Task 2}
\begin{itemize}
    \item[2.1]The first approach is to use the Composite design pattern to model the way directories are made up of files together with the Template method 
    pattern to model the difference between asynchronous directories and synchronous directories. The second approach is to use the Template method pattern 
    to model directories and files having shared attributes and the Strategy pattern to model the difference between asynchronous directories and synchronous
    directories. The first approach directly imitates the tree structure we want in our file system and therefore makes the operations to be applied to the system 
    more intuitive. The second approach is less intuitive since the tree structure of the program will have to be mimicked by other classes which adds extra 
    complexity to the system. 
    \item[2.2]I will follow the first approach by using the Composite design pattern to model the way directories are made up of files together with the Template
    method pattern to model the difference between asynchronous directories and synchronous directories.
\end{itemize}
\end{document}